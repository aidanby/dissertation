%%%%%%%%%%%%%%%%%%%%%%%%%%%%%%%%%%%%%%%%%%
% General
%%%%%%%%%%%%%%%%%%%%%%%%%%%%%%%%%%%%%%%%%%%

%% Comments
% \newcommand{\notes}[1]{}
% \newcommand{\TODO}[1]{}
\newif\ifshownotes
\shownotestrue
%\shownotesfalse

\ifshownotes
\newcommand{\notes}[1]{{\small \textsf{\textcolor{red}{[#1]}}}}
\newcommand{\TODO}[1]{\textcolor{red}{[TODO: #1]}}
\newcommand{\todo}[1]{\TODO{#1}}
\newcommand{\darion}[1]{\notes{Darion says: #1}}
\newcommand{\limin}[1]{\notes{Limin says: #1}}
\newcommand{\nuno}[1]{\notes{Nuno says: #1}}
\newcommand{\ruben}[1]{\notes{Ruben says: #1}}
\newcommand{\aidan}[1]{\notes{Aidan says: #1}}
\newcommand{\kevin}[1]{\notes{Kevin says: #1}}
\newcommand{\ronghao}[1]{\notes{Ronghao says: #1}}
\newcommand{\mindy}[1]{\notes{Mindy says: #1}}
\else
\newcommand{\notes}[1]{}
\newcommand{\TODO}[1]{}
\newcommand{\todo}[1]{}
\newcommand{\darion}[1]{}
\newcommand{\limin}[1]{}
\newcommand{\nuno}[1]{}
\newcommand{\ruben}[1]{}
\newcommand{\aidan}[1]{}
\newcommand{\kevin}[1]{}
\newcommand{\ronghao}[1]{}
\newcommand{\mindy}[1]{}
\fi



%% Formatting
\newcommand{\lilskip}{\vspace{.25\baselineskip}}

%% Boxes
\newcommand{\researchq}[2]{
\lilskip
\noindent\textbf{RQ#1}: #2
\lilskip
}
\newcommand{\researchqinline}[2]{\textbf{RQ#1}: #2}
\newcommand{\rqnum}[1]{\textbf{RQ#1}}

\newcommand{\resultbox}[2]{
\vspace{1mm}  \noindent\fbox{
    \begin{minipage}{0.96\columnwidth}
      \textbf{Result #1}: #2
    \end{minipage}%
  }%
  \vspace{1mm}
  }%

\newcommand{\textbox}[1]{
\vspace{1mm}  \noindent\fbox{
    \begin{minipage}{0.96\columnwidth}
      #1
    \end{minipage}%
  }%
}%

\newcommand{\headerbox}[2]{
\vspace{1mm}  \noindent\fbox{
    \begin{minipage}{0.96\columnwidth}
      \textbf{#1}: #2
    \end{minipage}%
  }%
}%

\newcommand{\codebox}[1]{\vspace{0.15cm} \\ #1 \vspace{0.15cm} \\}


%% References
\newcommand{\secref}[1]{Section~\ref{#1}}
\newcommand{\figref}[1]{Figure~\ref{#1}}
\newcommand{\tabref}[1]{Table~\ref{#1}}
\newcommand{\appref}[1]{Appendix~\ref{#1}}
\newcommand{\defref}[1]{Definition~\ref{#1}}
\newcommand{\algoref}[1]{Algorithm~\ref{#1}}
\newcommand{\chapref}[1]{Chapter~\ref{#1}}

%% Math
\newcommand{\m}[1]{\mathtt{#1}}

%% Theorem environments
% \newtheorem{theorem}{Theorem}
% \theoremstyle{plain}
% \newtheorem{thm}{Theorem}
% \newtheorem{lem}[thm]{Lemma}
% \newtheorem{cor}[thm]{Corollary}
% \newtheorem{defn}[thm]{Definition}
% \newtheorem{assumption}[thm]{Assume}
% \theoremstyle{definition}
% \newtheorem{definition}{Definition}[section]
% \newenvironment{proofsketch}{\noindent{\it Proof (sketch):}\hspace*{0.25em}}{ \hspace*{\fill} \qed}

%% Symbols
\makeatletter
\newsavebox{\@brx}
\newcommand{\llangle}[1][]{\savebox{\@brx}{\(\m@th{#1\langle}\)}%
  \mathopen{\copy\@brx\kern-0.5\wd\@brx\usebox{\@brx}}}
\newcommand{\rrangle}[1][]{\savebox{\@brx}{\(\m@th{#1\rangle}\)}%
  \mathclose{\copy\@brx\kern-0.5\wd\@brx\usebox{\@brx}}}
\makeatother


%%%%%%%%%%%%%%%%%%%%%%%%%%%%%%%%%%%%%%%%%%%
% NodeMedic
%%%%%%%%%%%%%%%%%%%%%%%%%%%%%%%%%%%%%%%%%%%

%% Acronyms
\newacronym{aci}{ACI}{Arbitrary Command Injection}
\newacronym{ace}{ACE}{Arbitrary Code Execution}
\newacronym{llms}{LLMs}{Large Language Models}

%% Project names
\newcommand{\nodemedic}{\textsc{NodeMedic}\xspace}
\newcommand{\nodemedicfine}{\textsc{NodeMedic-FINE}\xspace}
\newcommand{\name}{\textsc{NodeMedic-LM}\xspace}
\newcommand{\nameshort}{\textsc{NodeMedic-LM}\xspace}

%% Component names
\newcommand{\Enum}{Enumerator\xspace}

%% Node.js names
\newcommand{\nodejs}{{Node.js}\xspace}
\newcommand{\npmpkg}[1]{\emph{#1}\xspace}
\newcommand{\npm}{{npm}\xspace}

% Responsible disclosure of package names
% Comment for anonymization
% \newcommand{\cleartext}[1]{\npmpkg{#1}}
% \newcommand{\anontext}[1]{}
% Comment for de-anonymization
\newcommand{\cleartext}[1]{}
\newcommand{\anontext}[1]{\npmpkg{#1}}
\newcommand{\pkgname}[2]{\cleartext{#1}\anontext{#2}}

%% Symbols
\newcommand{\circled}[1]{\tikz[baseline=(char.base)]{\node[shape=circle,draw,inner sep=1.2pt] (char) {\scriptsize{#1}};}}

%% Code
%\lstinline[style=customjs]|exec|
\newcommand{\sink}[1]{\js{#1}}

%% Evaluation table circles
\newcommand*\emptycirc[1][1ex]{\tikz\draw[thick] (0,0) circle (#1);}
\newcommand*\halfcirc[1][1ex]{%
  \begin{tikzpicture}
  \draw[fill] (0,0)-- (90:#1) arc (90:270:#1) -- cycle ;
  \draw[thick] (0,0) circle (#1);
  \end{tikzpicture}}
\DeclareRobustCommand*\fullcirc[1][1ex]{\tikz\fill (0,0) circle (#1);}

%% space
\renewcommand{\paragraph}[1]{\vspace{0pt}\noindent{\bf #1}}

\usepackage{listings}
\usepackage{xcolor}
% \usepackage{inconsolata} % For a better monospaced font
\usepackage{tikz}
\usetikzlibrary{arrows.meta, shapes.geometric, positioning}

% Define JavaScript language for listings
\lstdefinelanguage{JavaScript}{
    keywords={function, var, require, if, else, return, try, catch},
    keywordstyle=\color{blue}\bfseries,
    ndkeywords={__set_taint__, exec},
    ndkeywordstyle=\color{orange}\bfseries,
    identifierstyle=\color{black},
    sensitive=false,
    comment=[l]{//},
    morecomment=[s]{/*}{*/},
    commentstyle=\color{gray}\itshape,
    stringstyle=\color{red},
    morestring=[b]',
    morestring=[b]"
}

% Settings for listings
\lstset{
    language=JavaScript,
    basicstyle=\ttfamily\footnotesize, % Use a better monospaced font
    breaklines=true,
    keywordstyle=\color{blue}\bfseries,
    stringstyle=\color{red},
    commentstyle=\color{gray}\itshape,
    numbers=left,                     % Line numbers on the left
    numberstyle=\tiny\color{gray},   % Line numbers styling
    frame=single,                    % Frame around the code
    captionpos=b,                    % Caption below the code
    xleftmargin=15pt,                % Margin on the left
    xrightmargin=15pt                % Margin on the right
}

\usepackage{pifont}
\newcommand{\cmark}{\checkmark} % Checkmark symbol
\newcommand{\xmark}{\ding{55}} % Cross symbol
